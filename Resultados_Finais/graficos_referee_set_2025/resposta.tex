\documentclass[a4paper,12pt]{article}
\usepackage{amsmath}
\usepackage{amssymb}

\begin{document}
	
	\section*{Hamiltoniano do Sistema Sierpinski com Desordem Anderson - $\beta=1$}
	
	O Hamiltoniano do sistema é dado por:
	
	\begin{equation}
		\begin{aligned}
			H &= \sum_{i \in \text{Sierpinski}} \epsilon_i \, c_i^\dagger c_i
			- \sum_{\langle i,j \rangle \in \text{Sierpinski}} t \, c_i^\dagger c_j, \\
			\epsilon_i &= W \left( \text{uniforme}[-0.5, 0.5] \right) \, \sigma_0, \\
			t &= \sigma_0,
		\end{aligned}
	\end{equation}
	
	\noindent onde:
	
	\begin{itemize}
		\item $c_i^\dagger = \begin{pmatrix} c_{i\uparrow}^\dagger & c_{i\downarrow}^\dagger \end{pmatrix}$ é o operador de criação no site $i$ para os dois spinors.
		\item $\sigma_0 = \begin{pmatrix} 1 & 0 \\ 0 & 1 \end{pmatrix}$ é a matriz identidade $2 \times 2$ (spin-degenerado).
		\item $i \in \text{Sierpinski}$ indica que apenas os sites restantes após a construção fractal são considerados.
		\item $\langle i,j \rangle$ indica pares de sites vizinhos conectados pelo hopping $t$.
		\item $W$ é a intensidade da desordem Anderson.
	\end{itemize}
	
	
	
	
	\section*{Hamiltoniano do Sistema Sierpinski com Desordem Anderson - $\beta=2$}
	
	
	O Hamiltoniano do sistema é dado por:
	
	\begin{equation}
		\begin{aligned}
			H &= \sum_{i \in \text{Sierpinski}} \epsilon_i \, c_i^\dagger c_i
			- \sum_{\langle i,j \rangle \in \text{Sierpinski}} t_{ij} \, c_i^\dagger c_j, \\
			\epsilon_i &= W \left( \text{uniforme}[-0.5, 0.5] \right) \, \sigma_0, \\
			t_{ij} &= \sigma_0 \, e^{ i \frac{\phi}{2} (x_i - x_j)(y_i + y_j) },
		\end{aligned}
	\end{equation}
	
	\noindent onde:
	
	\begin{itemize}
		\item $c_i^\dagger = \begin{pmatrix} c_{i\uparrow}^\dagger & c_{i\downarrow}^\dagger \end{pmatrix}$ é o operador de criação no site $i$ para os dois spinors.
		\item $\sigma_0 = \begin{pmatrix} 1 & 0 \\ 0 & 1 \end{pmatrix}$ é a matriz identidade $2 \times 2$ (spin-degenerado).
		\item $i \in \text{Sierpinski}$ indica que apenas os sites restantes após a construção fractal são considerados.
		\item $\langle i,j \rangle$ indica pares de sites vizinhos conectados pelo hopping $t_{ij}$.
		\item $W$ é a intensidade da desordem Anderson.
		\item $\phi$ é a fase de Peierls, responsável por simular efeitos de campo magnético ou fluxos.
	\end{itemize}
	
	
	
	
	
	\section*{Hamiltoniano do Sistema Sierpinski com Desordem, Spin-Orbit e Campo Zeeman}
	
	O Hamiltoniano do sistema é dado por:
	
	\begin{equation}
		\begin{aligned}
			H &= \sum_{i \in \text{Sierpinski}} \left( \epsilon_i + e_z \, \sigma_z \right) c_i^\dagger c_i
			- \sum_{\langle i,j \rangle_x} \left( t \, \sigma_0 - \frac{i \alpha}{2} \sigma_y \right) c_i^\dagger c_j
			- \sum_{\langle i,j \rangle_y} \left( t \, \sigma_0 + \frac{i \alpha}{2} \sigma_x \right) c_i^\dagger c_j, \\
			\epsilon_i &= W \left( \text{uniforme}[-0.5, 0.5] \right)
		\end{aligned}
	\end{equation}
	
	\noindent onde:
	
	\begin{itemize}
		\item $c_i^\dagger = \begin{pmatrix} c_{i\uparrow}^\dagger & c_{i\downarrow}^\dagger \end{pmatrix}$ é o operador de criação no site $i$.
		\item $\sigma_0, \sigma_x, \sigma_y, \sigma_z$ são as matrizes de Pauli.
		\item $i \in \text{Sierpinski}$ indica que apenas os sites restantes após a construção fractal são considerados.
		\item $\langle i,j \rangle_x$ e $\langle i,j \rangle_y$ indicam pares de sites vizinhos nas direções $x$ e $y$, respectivamente.
		\item $W$ é a intensidade da desordem Anderson.
		\item $e_z$ é o termo Zeeman (campo magnético).
		\item $\alpha$ é a intensidade do spin-orbit Rashba.
	\end{itemize}
	
	
	
	
	
	
\end{document}
